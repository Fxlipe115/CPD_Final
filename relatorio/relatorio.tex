\documentclass[12pt]{article}

\usepackage{sbc-template}

\usepackage{graphicx,url}

\usepackage[brazil]{babel}
\usepackage[utf8]{inputenc}


\sloppy

\title{Trabalho final de Classificação e Pesquisa de Dados}

\author{Arthur Zachow\inst{1} e Felipe de Almeida Graeff\inst{1}}


\address{Instituto de Inform�tica -- Universidade Federal do Rio Grande do Sul
  (UFRGS)\\
  Caixa Postal 15.064 -- 91.501-970 -- Porto Alegre -- RS -- Brazil
  \email{\{azcoelho, fagraeff\}@inf.ufrgs.br}
}

\begin{document}

\maketitle

\section{Introdu��o}
O software desenvolvido para o trabalho final da cadeira de Classifica��o e Pesquisa de Dados � um
analizador de sentimentos de coment�rios sobre filmes. O fato de ter sido desenvolvido com o
objetivo de ser utilizado para a avalia��o de coment�rios de filmes n�o impede de forma alguma a
utiliza��o para a classifica��o de outras coisas, como por exemplo: Tweets e coment�rios do
Facebook, desde que seja fornecida corretamente a entrada da forma especificada mais adiante no
texto.

O funcionamento do programa e a classificação dos comentários depende grandemente do arquivo de
entrada fornecido, pois ele é usado como base para os cálculos de pontuação.

O sistema de pontuação é o intervalo de 0--4 nos inteiros. Com 0 representando um sentimento "Muito
negativo"; 1 sendo um sentimento "Negativo"; 2 representa um sentimento "Neutro"; 3 codifica um
sentimento "Positivo"; 4 é classificado como um sentimento "Muito positivo".

\section{Desenvolvimento}
O software foi desenvolvido na linguagem C++, padrão C++11, com o auxílio de poucas estruturas de
dados nativas, apenas as classes "vector" e "list" foram utilizadas al�m dos tipos de dados b�sicos
(string, int, double, etc...), pois esta era uma das limitações impostas na descrição do
trabalho.

As funções foram desenvolvidas e alocadas em arquivos separados da forma que os desenvolvedores
julgaram conveniente no momento. Em 5 arquivos temos na pasta "src" temos: "file\_functions.cpp", o
qual é a biblioteca a qual possui a implementação das funções diretamente relacionadas com a leitura
do arquivo de entrada; "hash\_table.cpp" onde está a implementação da classe Hash Table (auto
explicativo); "main.cpp" onde está a função main do programa e a implementação dos menus e coisas
feitas diretamente para a interatividade do usuário; "review.cpp" é a implementação da classe Review
que representa um comentário; "word.cpp" é a implementação da classe Word, a qual representa uma
palavra a qual aparece ao menos em um comentário. A leitura do código pode ser feita
"https://github.com/Fxlipe115/CPD\_Final" é encorajada pois ele não será apontado diretamente na
descrição da maior parte do que foi implementado.

A classe Word possui os seguintes atributos: sum que � a soma dos valores da palavra em todos os
reviews; key que é a string que é a palavra em si; occurrences que é a quantidade de ocorrências da
palavra; pos é a quantidade de vezes na qual a pontuação da palavra foi >= 2; neg é a quantidade de
vezes que a pontuação foi < 2; reviews é uma lista de inteiros que são ponteiros para entradas na
hash\_table dos reviews onde essa palavra está presente. Os métodos fora getters, setters e
incrementadores de 1 nos atributos são apenas 2 métodos que são: wil\_lower\_bound, qual é o valor
inferior do intervalo de confiança de Wilson; mean é o método que segue o algoritmo a seguir para o
cálculo da pontuação da palavra.

O algoritmo de cálculo da pontuação da palavra é: primeiramente é lido o arquivo de entrada por
completo e então utilizando os atributos da classe Word é calculado a média simples, após isso esse
valor calculado é feita a diferença para o valor neutro (2), então o resultado dessa operação é
utilizado para multiplicar pelo valor do intervalo de confiança de Wilson e finalmente a esse último
valor é somado 2 para retornar ao intervalo desejado [0--4].

$$ \frac{1}{1+\frac{1}{n}z^2}\left [ \hat{p} + \frac{1}{2n}z^2 \pm z \sqrt{\frac{1}{n} \hat{p}
   \left ( 1 - \hat{p} \right ) + \frac{1}{4n^2} z^2} \right ] $$

\section{Conclusion}

\end{document}
